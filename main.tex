\documentclass[9pt]{article}
\usepackage{graphicx} % Required for inserting images

\title{RLBasics}
\author{sagarsrinivastcs }
\date{October 2025}
\usepackage{amsmath, amssymb}
\usepackage{algorithm}    % Algorithm environment
\usepackage{algpseudocode}% Pseudocode environment for algorithms
\usepackage{mdframed}

\begin{document}

\begin{flushleft}
\section*{What Is Model-Based Reinforcement Learning (MBRL)?}

Model-based reinforcement learning (MBRL) is a family of RL methods where the agent learns an \emph{explicit} model of the environment's dynamics and uses that model for planning or improving its policy, rather than relying solely on trial-and-error interaction as in model-free RL.

\section{Core Idea}

In a Markov Decision Process (MDP), the environment is defined by:

\begin{itemize}
    \item \textbf{State transition dynamics}
    \[
        p(s_{t+1} \mid s_t, a_t)
    \]
    \item \textbf{Reward function}
    \[
        r(s_t, a_t)
    \]
\end{itemize}

Model-based RL tries to learn approximations
\[
\hat{p}_\theta(s_{t+1} \mid s_t, a_t), 
\qquad 
\hat{r}_\phi(s_t, a_t),
\]
from data, and then uses them for planning or policy learning.  
Model-free RL skips this step and directly learns a policy or Q-function.

\section{The Three Pillars of Model-Based RL}

\subsection{Learning the Model}

You learn predictive functions from collected transitions:
\[
(s_t, a_t, r_{t+1}, s_{t+1}) \sim \mathcal{D}.
\]

\begin{itemize}
    \item \textbf{Forward dynamics model} (predicts next state)
    \[
        \hat{s}_{t+1} = f_\theta(s_t, a_t)
    \]
    \item \textbf{Reward model} (predicts reward)
    \[
        \hat{r}_{t+1} = g_\phi(s_t, a_t)
    \]
    \item \textbf{Terminal/done predictor} (optional)
    \[
        \hat{d}_{t+1} = h_\psi(s_t, a_t)
    \]
\end{itemize}

\subsection{Planning Using the Model}

Use the learned model to simulate future trajectories.  
Standard choices include:

\begin{itemize}
    \item \textbf{MPC (Model Predictive Control)}  
    Roll out candidate action sequences through the model and choose the best via an optimizer (e.g., CEM, MPPI).

    \item \textbf{Value expansion}  
    Use short model rollouts combined with bootstrapping (e.g., STEVE in MBPO).

    \item \textbf{Policy optimization inside model rollouts}  
    Update the policy using trajectories imagined from the learned model.
\end{itemize}

\subsection{Improving the Policy}

Using simulated rollouts, you may:
\begin{itemize}
    \item update Q-functions,
    \item update policy gradients,
    \item reduce sample complexity,
    \item reduce unsafe exploration.
\end{itemize}

A typical policy update:
\[
\theta \leftarrow 
\theta 
- 
\eta \nabla_\theta 
\mathbb{E}_{\hat{\tau} \sim \hat{p}} 
\left[
    \sum_t r(s_t, a_t)
\right].
\]
\end{flushleft}

\newpage
\pagebreak


\begin{flushleft}
\section{PETS: Probabilistic Ensembles with Trajectory Sampling}

We consider a continuous-state, continuous-action MDP
\[
\mathcal{M} = (\mathcal{S},\mathcal{A},p,r,\gamma),
\]
where the dynamics $p(s_{t+1}\mid s_t,a_t)$ are unknown and must be learned.

\subsection{Notation}
\begin{itemize}
    \item $s_t \in \mathcal{S} \subseteq \mathbb{R}^{d_s}$: state at time $t$.
    \item $a_t \in \mathcal{A} \subseteq \mathbb{R}^{d_a}$: action at time $t$.
    \item $r(s_t,a_t)$: deterministic reward.
    \item $H$: MPC planning horizon.
    \item $B$: number of models in the ensemble.
    \item $N$: number of particles used for trajectory prediction.
    \item $K$: number of candidate action sequences sampled per CEM iteration.
    \item $K_{\mathrm{elite}}$: number of top candidates retained.
    \item $J$: number of CEM iterations.
    \item $\alpha \in [0,1]$: smoothing coefficient for CEM parameter updates.
\end{itemize}

\section{Probabilistic Ensemble Dynamics}
PETS learns an ensemble of $B$ probabilistic neural dynamics models, each
parameterized by $\theta^{(b)}$. Each model predicts a Gaussian transition distribution for the next state. These models are trained using \emph{real} environment transitions collected so far.

\subsection*{Dataset and Training Setup}
PETS maintains a dataset of real transitions:
\[
\mathcal{D}
=
\{(s_t,a_t,r_{t+1},s_{t+1})\}.
\]

\begin{itemize}
    \item This dataset contains \textbf{only real environment data}, \emph{not} model-generated rollouts.
    \item Every ensemble model is trained on a bootstrap resample of this same dataset.
\end{itemize}

PETS typically predicts \emph{state differences},
\[
s_{t+1} = s_t + f_{\theta^{(b)}}(s_t,a_t),
\]
but for clarity we use absolute-state prediction. This does not affect the
derivations.

\subsection*{Model Parameterization}

\begin{itemize}
    \item Each model defines the conditional transition distribution
    \[
    p_{\theta^{(b)}}(s_{t+1}\mid s_t,a_t)
    =
    \mathcal{N}\!\Big(
        \mu_{\theta^{(b)}}(s_t,a_t),\;
        \Sigma_{\theta^{(b)}}(s_t,a_t)
    \Big),
    \qquad \forall b\in\{1,\dots,B\}.
    \]

    \item The neural network outputs:
    \begin{itemize}
        \item mean: $\mu_{\theta^{(b)}}(s,a)$,
        \item log-standard-deviation: $\log {\sigma}_{\theta^{(b)}}(s,a)$.
    \end{itemize}

    \item Standard deviations are recovered as
    \[
    \sigma_{\theta^{(b)}}(s,a)
    =
    \exp\!\big( \log {\sigma}_{\theta^{(b)}}(s,a) \big),
    \]
    \item The diagonal covariance is therefore
    \[
    \Sigma_{\theta^{(b)}}(s,a)
    =
    \mathrm{diag}\!\left(
        \sigma_{\theta^{(b)},1}^2(s,a),\dots,
        \sigma_{\theta^{(b)},d_s}^2(s,a)
    \right).
    \]
\end{itemize}

\subsection*{Sampling via Reparameterization}

\begin{itemize}
    \item PETS generates samples using the reparameterization rule.  
    For $\epsilon \sim \mathcal{N}(0,I_{d_s})$:
    \[
    s_{t+1}^{(b)}
    =
    \mu_{\theta^{(b)}}(s_t,a_t)
    +
    \sigma_{\theta^{(b)}}(s_t,a_t)\odot \epsilon.
    \]

    \item This sampling expression is equivalent to drawing from the Gaussian
    \[
    \mathcal{N}\!\big(
        \mu_{\theta^{(b)}}(s_t,a_t),\;
        \mathrm{diag}(\sigma_{\theta^{(b)}}^2(s_t,a_t))
    \big),
    \]
    but represents a single realization rather than the distribution itself.
\end{itemize}

\subsection{Training Objective}

Given a dataset 
\(
\mathcal{D}=\{(s_n,a_n,s_{n+1})\}_{n=1}^{|\mathcal{D}|},
\)
each ensemble model $b$ is trained on a bootstrap resample 
$\mathcal{D}_b$ by minimizing the negative log-likelihood of the
Gaussian transition model:
\[
\mathcal{L}(\theta^{(b)})
=
\sum_{n\in\mathcal{D}_b}
\left[
\big(s_{n+1}-\mu_{\theta^{(b)}}(s_n,a_n)\big)^\top
\Sigma_{\theta^{(b)}}^{-1}(s_n,a_n)
\big(s_{n+1}-\mu_{\theta^{(b)}}(s_n,a_n)\big)
+
\log\det \Sigma_{\theta^{(b)}}(s_n,a_n)
\right].
\]

This updates the model parameters $\theta^{(1)}, \ldots, \theta^{(B)}$.  A trained PETS model $b$ predicts a probabilistic next-state transition by computing the mean $\mu_{\theta^{(b)}}(s_t,a_t)$ and covariance $\Sigma_{\theta^{(b)}}(s_t,a_t)$, with samples generated via
\[
s_{t+1}^{(b)}
=
\mu_{\theta^{(b)}}(s_t,a_t)
+
\sigma_{\theta^{(b)}}(s_t,a_t)\odot\epsilon,
\qquad
\epsilon\sim\mathcal{N}(0,I).
\]

\begin{itemize}
    \item PETS uses Gaussian dynamics models whose predicted variances represent aleatoric uncertainty.
    \item PETS employs bootstrap-resampled datasets to induce model diversity, representing epistemic uncertainty.
    \item PETS trains each model by minimizing the Gaussian negative log-likelihood.
\end{itemize}

\begin{mdframed}[linecolor=red, linewidth=2pt]
\subsection{Derivation of the PETS Negative Log-Likelihood Loss}
\paragraph{Step 1: Multivariate Gaussian Transition Model.}
For each transition $(s_n,a_n,s_{n+1})$, the ensemble model $b$ predicts
\[
p_{\theta^{(b)}}(s_{n+1}\mid s_n,a_n)
=
\mathcal{N}\!\Big(
\mu_{\theta^{(b)}}(s_n,a_n),\;
\Sigma_{\theta^{(b)}}(s_n,a_n)
\Big),
\]
where $\mu_{\theta^{(b)}}(s_n,a_n)\in\mathbb{R}^{d_s}$ and
$\Sigma_{\theta^{(b)}}(s_n,a_n)\in\mathbb{R}^{d_s\times d_s}$.

\paragraph{Step 2: Negative Log-Likelihood of a Multivariate Gaussian.}
The probability density of a $d_s$-dimensional Gaussian is
\[
p(x\mid\mu,\Sigma)
=
\frac{1}{(2\pi)^{d_s/2}|\Sigma|^{1/2}}
\exp\!\left(
-\tfrac{1}{2}(x-\mu)^\top \Sigma^{-1}(x-\mu)
\right).
\]
Taking $x=s_{n+1}$, the negative log-likelihood becomes
\[
-\log p(s_{n+1})
=
\tfrac{1}{2}(s_{n+1}-\mu)^\top \Sigma^{-1}(s_{n+1}-\mu)
+
\tfrac{1}{2}\log\det\Sigma
+
\tfrac{d_s}{2}\log(2\pi).
\]

\paragraph{Step 3: Removing Constants (PETS Approximation).}
Because the constant $\frac{d_s}{2}\log(2\pi)$ and the factor
$\frac{1}{2}$ do not depend on $\theta^{(b)}$, PETS removes them:
\[
\mathcal{L}(s_{n+1})
=
(s_{n+1}-\mu)^\top\Sigma^{-1}(s_{n+1}-\mu)
+
\log\det\Sigma.
\]

\paragraph{Step 4: Diagonal Covariance Structure.}
\[
\Sigma_{\theta^{(b)}}(s_n,a_n)
=
\mathrm{diag}\!\left(\sigma_1^2,\dots,\sigma_{d_s}^2\right),
\qquad
\Sigma^{-1}_{\theta^{(b)}}
=
\mathrm{diag}\!\left(\sigma_1^{-2},\dots,\sigma_{d_s}^{-2}\right).
\]

\[
(s_{n+1}-\mu)^\top \Sigma^{-1}(s_{n+1}-\mu)
=
\sum_{i=1}^{d_s} \frac{(s_{n+1,i}-\mu_i)^2}{\sigma_i^2}.
\]

\[
\log\det\Sigma
=
\sum_{i=1}^{d_s} \log(\sigma_i^2)
=
2\sum_{i=1}^{d_s} \log\sigma_i.
\]

\paragraph{Step 5: Neural Network Parameterization.}
Since
\[
\sigma_i = \exp(\log \tilde{\sigma}_i),
\qquad
\log \sigma_i = \log \tilde{\sigma}_i,
\]
the log-determinant becomes
\[
\log\det\Sigma = 2 \sum_{i=1}^{d_s} \log\tilde{\sigma}_i.
\]

\paragraph{Step 6: Final PETS Loss.}
\[
\begin{split}
\mathcal{L}(\theta^{(b)}) = \sum_{n\in\mathcal{D}_b} \bigg[ &
\big(s_{n+1}-\mu_{\theta^{(b)}}(s_n,a_n)\big)^\top 
\Sigma_{\theta^{(b)}}^{-1}(s_n,a_n) 
\big(s_{n+1}-\mu_{\theta^{(b)}}(s_n,a_n)\big)  \\
& + \log\det \Sigma_{\theta^{(b)}}(s_n,a_n) \bigg].
\end{split}
\]
\end{mdframed}

\begin{mdframed}[linecolor=red, linewidth=2pt]
\subsection{Bootstrap Resampling in PETS}

PETS uses bootstrap ensembles to capture epistemic uncertainty.

\begin{itemize}
    \item For each ensemble model $b \in \{1,\dots,B\}$:
    \[
    \mathcal{D}_b = \{(s_{n_i},a_{n_i},s_{n_i+1})\}_{i=1}^{|\mathcal{D}|},
    \]
    where indices $n_i$ are sampled \emph{with replacement}.
\end{itemize}

\[
\mathrm{Var}_b\!\left[
p_{\theta^{(b)}}(s_{t+1}\mid s_t,a_t)
\right]
\]
quantifies epistemic uncertainty.
\end{mdframed}

\section{PETS Phase 2: Trajectory Sampling (TS$\infty$) and MPC Planning with CEM}

Once the ensemble dynamics models $\{\theta^{(1)},\dots,\theta^{(B)}\}$ are 
trained (Phase~1), their parameters are \emph{frozen}. PETS then performs 
model-predictive control by optimizing over action sequences rather than model 
parameters. Phase~2 consists of two components: (i) Trajectory Sampling under 
persistent model assignments (TS$\infty$), and (ii) action-sequence 
optimization using the Cross-Entropy Method (CEM).

\subsection{Step 1: Persistent Model Assignment (TS$\infty$)}
To account for epistemic uncertainty, PETS assigns each particle to a single
ensemble model for the entire planning horizon $H$:
\[
b_i \sim \mathrm{Uniform}\{1,\dots,B\},
\qquad i=1,\dots,N.
\]
Each particle $i$ therefore evolves using model $b_i$ for all predicted future 
transitions.

\begin{itemize}

    \item \textbf{Candidate action sequence $A$.}
    CEM proposes a length-$H$ sequence
    \[
        A = (a_t, a_{t+1}, \dots, a_{t+H-1}),
    \]
    and all actions used during rollout satisfy
    \[
        a_{t+h} \;\text{is the $h$-th element of } A,\qquad h=0,\dots,H-1.
    \]

    \item \textbf{Particles (parallel stochastic simulations).}
    For each candidate $A$, PETS evaluates it using $N$ particles.
    Particles correspond to \emph{multiple stochastic rollouts of the same}
    sequence $A$.  
    Each particle:
    \begin{itemize}
        \item samples one ensemble model index $b_i$,
        \item applies the same action sequence $A$,
        \item generates a trajectory
        \[
            s_t^{(i)} \rightarrow s_{t+1}^{(i)}
            \rightarrow \dots \rightarrow
            s_{t+H}^{(i)}.
        \]
    \end{itemize}

    \item \textbf{Particle state propagation.}
    Using the action $a_{t+h}$ taken from $A$:
    \[
        s_{t+h+1}^{(i)}
        =
        \mu_{\theta^{(b_i)}}\!\left(s_{t+h}^{(i)}, a_{t+h}\right)
        +
        \xi_{t+h}^{(i)},\qquad
        \xi_{t+h}^{(i)}\sim
        \mathcal{N}\!\Big(0,
        \Sigma_{\theta^{(b_i)}}\!\left(s_{t+h}^{(i)}, a_{t+h}\right)\Big).
    \]

    \item \textbf{Outcome of particle rollout.}
    After propagating states for all $h=0,\dots,H-1$, particle $i$ produces
    the full sequence of state--action pairs
    \[
        (s_t^{(i)}, a_t),\;
        (s_{t+1}^{(i)}, a_{t+1}),\;
        \dots,\;
        (s_{t+H-1}^{(i)}, a_{t+H-1}),
    \]
    together with the terminal state $s_{t+H}^{(i)}$.

    \item \textbf{Return estimation of $A$.}
    \[
        \hat{J}(A)
        =
        \frac{1}{N}
        \sum_{i=1}^N
        \sum_{h=0}^{H-1}
        \gamma^{h}\,
        r\!\left(s_{t+h}^{(i)}, a_{t+h}\right).
    \]

    \item \textbf{Why PETS Does Not Model the Reward.}
    PETS assumes that the environment already provides a known reward function
    \[
        r(s_t, a_t),
    \]
    and therefore does not learn a separate reward model. 

\end{itemize}

\subsection{Step 2: Particle Initialization}
All particles begin from the current real environment state:
\[
s_t^{(i)} = s_t, 
\qquad i = 1,\dots,N.
\]

\subsection{Step 3: Candidate Action Sequence Sampling}
CEM maintains a Gaussian search distribution over length-$H$ action sequences:
\[
A = (a_t, a_{t+1}, \dots, a_{t+H-1})
\sim \mathcal{N}\!\left(\mathbf{m}^{(j)},\,\mathbf{C}^{(j)}\right),
\]
where $\mathbf{m}^{(j)} \in \mathbb{R}^{H d_a}$ is the mean vector and 
$\mathbf{C}^{(j)} \in \mathbb{R}^{H d_a \times H d_a}$ is typically diagonal.

Sampling uses a valid reparameterization:
\[
A = \mathbf{m}^{(j)} + L^{(j)} \epsilon,
\qquad
\epsilon \sim \mathcal{N}(0, I_{Hd_a}),
\qquad
L^{(j)} = \mathrm{diag}(\sigma^{(j)}),
\]
where $\sigma^{(j)}$ is the vector of per-dimension standard deviations. $L^{(j)}$ is the Cholesky factor of $C^{(j)}$,  $\qquad C^{(j)} = L^{(j)} {L^{(j)}}^{\!\top}$.

\begin{itemize}
    \item \textbf{Action dimension.}  
    Each action is a $d_a$-dimensional continuous vector:
    \[
       a_t \in \mathbb{R}^{d_a}.
    \]
\end{itemize}

\subsection{Step 4: Trajectory Rollouts Under Model $b_i$}
For each sampled candidate sequence $A$ and particle $i$, PETS propagates the
state forward using the model permanently assigned to the particle:
\[
s_{t+h+1}^{(i)}
=
\mu_{\theta^{(b_i)}}\!\left(s_{t+h}^{(i)},\,a_{t+h}\right)
+
\sigma_{\theta^{(b_i)}}\!\left(s_{t+h}^{(i)},\,a_{t+h}\right)
\odot \epsilon_{t+h}^{(i)},
\]
\[
\epsilon_{t+h}^{(i)} \sim \mathcal{N}(0,I_{d_s}),
\qquad h = 0,\dots,H-1.
\]

This captures:
\begin{itemize}
    \item \textbf{aleatoric uncertainty} via Gaussian noise, and
    \item \textbf{epistemic uncertainty} via persistent model assignment $b_i$.
\end{itemize}

\subsection{Step 5: Monte Carlo Return Estimation}
The predicted return of the action sequence $A$ is estimated by averaging 
particle rollouts:
\[
\hat{J}(A)
=
\frac{1}{N}
\sum_{i=1}^{N}
\sum_{h=0}^{H-1}
\gamma^{h}\,
r\!\left(s_{t+h}^{(i)},\,a_{t+h}\right).
\]

\subsection{Step 6: Elite Selection in CEM}
After evaluating $K$ candidate sequences 
$\{A^{(1)},\dots,A^{(K)}\}$, the top $K_{\mathrm{elite}}$ sequences (highest 
predicted returns) are chosen:
\[
\mathcal{E}^{(j)} =
\left\{ 
A^{(k_1)},\dots,A^{(k_{K_{\mathrm{elite}}})}
\right\},
\qquad
\hat{J}\!\left(A^{(k_1)}\right)
\ge \dots \ge
\hat{J}\!\left(A^{(k_{K_{\mathrm{elite}}})}\right).
\]

\subsection{Step 7: Update CEM Mean and Covariance}
Elite mean:
\[
\mathbf{m}_{\mathrm{new}}
=
\frac{1}{K_{\mathrm{elite}}}
\sum_{A\in\mathcal{E}^{(j)}} A.
\]

Elite covariance:
\[
\mathbf{C}_{\mathrm{new}}
=
\frac{1}{K_{\mathrm{elite}}}
\sum_{A\in\mathcal{E}^{(j)}}
\left(A-\mathbf{m}_{\mathrm{new}}\right)
\left(A-\mathbf{m}_{\mathrm{new}}\right)^\top.
\]

Exponential smoothing to prevent premature collapse:
\[
\mathbf{m}^{(j+1)}
=
\alpha\,\mathbf{m}_{\mathrm{new}}
+
(1-\alpha)\,\mathbf{m}^{(j)},
\]
\[
\mathbf{C}^{(j+1)}
=
\alpha\,\mathbf{C}_{\mathrm{new}}
+
(1-\alpha)\,\mathbf{C}^{(j)}.
\]

\subsection{Step 8: Final Action Selection}
After $J$ iterations of CEM, the optimized mean sequence
$\mathbf{m}^{(J)}$ defines the MPC plan. The first action is applied to the
real environment:
\[
a_t^{*} = \mathbf{m}^{(J)}_1.
\]

\end{flushleft}

\newpage
\pagebreak

\section*{1. PETS Planning Modes}

PETS supports two distinct ways of generating action sequences:

\subsection*{(A) Trajectory Sampling (TS$\infty$)}
\textit{(Standalone planning mode --- no optimization)}

\begin{itemize}
    \item Pick one dynamics model per particle.
    \item Sample many action sequences from a fixed distribution.
    \item For each sequence, roll out TS$\infty$ particles to compute predicted return
    (each particle uses its assigned model for the entire horizon).
    \item Choose the best action sequence based purely on return.
    \item \textbf{No optimization step.}
    \item \textbf{No elite selection.}
    \item \textbf{No mean/variance updates.}
\end{itemize}

\subsection*{(B) CEM-MPC (Cross-Entropy Method)}
\textit{(Optimization-based planning mode)}

\begin{itemize}
    \item Maintain a Gaussian distribution over action sequences.
    \item Sample a batch of candidate action sequences from this Gaussian.
    \item \textbf{Evaluate each sequence using TS$\infty$ rollouts} 
    (TS$\infty$ is used here as the return estimator).
    \item Rank sequences by predicted return.
    \item Select the top-$K$ elites.
    \item Update the mean and covariance of the Gaussian distribution.
    \item Repeat for $J$ iterations to refine the distribution.
    \item Output the final optimized action sequence.
\end{itemize}

\noindent\textbf{Key insight:} TS$\infty$ is not a separate planning mode inside CEM; it is the \emph{inner-loop return evaluator} used during CEM optimization.

\subsection*{2.1.\hspace{0.2cm}PETS — Probabilistic Ensembles with Trajectory Sampling}

PETS is a model-based RL framework that includes:
\begin{itemize}
    \item A probabilistic ensemble dynamics model
    \item A trajectory sampling mechanism (TS$\infty$)
    \item A planning algorithm to choose actions
\end{itemize}

Inside PETS, action selection can be done by:
\begin{itemize}
    \item CEM-MPC
    \item Shooting (random shooting)
    \item MPPI
    \item Or any other planning method
\end{itemize}

Thus,
\[
\text{PETS} = \text{Ensemble dynamics} + \text{Trajectory sampling} + \text{Planner}.
\]

\subsection*{3.\hspace{0.2cm}Model Predictive Control (MPC)}

MPC itself is just a planning technique:
\begin{itemize}
    \item Roll out candidate action sequences
    \item Evaluate predicted returns using the model
    \item Pick the best sequence (or optimize it)
\end{itemize}

MPC does not specify what the model is; it only tells you \emph{how} to plan once a model exists.

\subsection*{3.1.\hspace{0.2cm}CEM-MPC}

CEM-MPC is simply:
\begin{itemize}
    \item MPC whose optimizer is the Cross-Entropy Method (CEM)
    \item Applied to the learned model to produce an optimized action sequence
\end{itemize}

It is one type of planner used inside PETS.



\section{Algorithm: PETS (Probabilistic Ensembles with Trajectory Sampling)}

\begin{algorithm}[H]
\caption{PETS: Probabilistic Ensembles with Trajectory Sampling (Part 1)}
\begin{algorithmic}[1]

\State \textbf{Inputs:}
\State \hspace{0.5cm} Ensemble size $B$
\State \hspace{0.5cm} Planning horizon $H$
\State \hspace{0.5cm} Number of particles $N$
\State \hspace{0.5cm} Number of CEM samples $K$
\State \hspace{0.5cm} Number of elite samples $K_{\text{elite}}$
\State \hspace{0.5cm} Number of CEM iterations $J$
\State \hspace{0.5cm} Smoothing factor $\alpha$
\State \hspace{0.5cm} Dataset $\mathcal{D}$ initially empty

\State Initialize ensemble model parameters $\theta^{(1)}, \dots, \theta^{(B)}$
\State Initialize CEM mean vector and covariance matrix

% ----------------------------------------------------------
\Statex \hrulefill
\Statex \textbf{Phase 1: Dataset Collection and Ensemble Training}
\Statex \hrulefill

\While{size of $\mathcal{D}$ is less than required initial samples}
    \State Select a random exploratory action $a_t$
    \State Execute $a_t$ in the real environment
    \State Observe next state $s_{t+1}$ and reward $r_{t+1}$
    \State Add tuple $(s_t, a_t, r_{t+1}, s_{t+1})$ to dataset $\mathcal{D}$
\EndWhile

\State \textbf{Train Ensemble Models:}
\For{each model index $b = 1$ to $B$}

    \State Construct bootstrap dataset $\mathcal{D}_b$ by sampling from $\mathcal{D}$ with replacement

    \State Each model predicts Gaussian dynamics:
    \State \hspace{0.7cm}
      $p_{\theta^{(b)}}(s' \mid s,a) = 
      \mathcal{N}\!\big(\mu_{\theta^{(b)}}(s,a),\;
                       \Sigma_{\theta^{(b)}}(s,a)\big)$

    \State Train model $b$ by minimizing negative log-likelihood:
    \State \hspace{0.7cm}
      $s_{n+1} \approx 
      \mu_{\theta^{(b)}}(s_n,a_n)
      + \sigma_{\theta^{(b)}}(s_n,a_n)\odot \epsilon, \;
      \epsilon \sim \mathcal{N}(0,I)$

    \State Prediction error term:
    \State \hspace{0.7cm}
      $(s_{n+1}-\mu_{\theta^{(b)}}(s_n,a_n))^\top 
       \Sigma_{\theta^{(b)}}^{-1}(s_n,a_n)
       (s_{n+1}-\mu_{\theta^{(b)}}(s_n,a_n))$

    \State Variance penalty:
    \State \hspace{0.7cm}
      $\log\det \Sigma_{\theta^{(b)}}(s_n,a_n)$

    \State Full loss for model $b$:
    \State \hspace{0.7cm}
      $\mathcal{L}(\theta^{(b)}) =
      \sum_{(s_n,a_n,s_{n+1}) \in \mathcal{D}_b}
      \Big[
      (s_{n+1}-\mu)^T \Sigma^{-1} (s_{n+1}-\mu)
      + \log\det\Sigma
      \Big]$

    \State Update $\theta^{(b)}$ via gradient descent

\EndFor

% Stop here and split pages
\algstore{pets}

\end{algorithmic}
\end{algorithm}


\begin{algorithm}[H]
\caption{PETS: Probabilistic Ensembles with Trajectory Sampling (Part 2)}
\begin{algorithmic}[1]
\algrestore{pets}

% ----------------------------------------------------------
\Statex \hrulefill
\Statex \textbf{Phase 2: MPC Action Selection using TS$\infty$ + CEM}
\Statex \hrulefill

\While{task not terminated}

    \State Observe current state $s_t$

    % ---------------------------------------------
    \Statex \textbf{CEM Optimization Loop}
    \For{$j = 0$ to $J-1$}

        \State Sample $K$ candidate action sequences ($A \sim \mathcal{N}\!\big(m^{(j)},\, C^{(j)}\big)$.
        \Statex \hspace{0.7cm} Each sequence sampled as:
        \Statex \hspace{0.7cm} $A = m^{(j)} + L^{(j)}\epsilon,\;\; \epsilon \sim \mathcal{N}(0,I)$

        % ---------------------------------------------
        \Statex \textbf{Evaluate Each Candidate Sequence}
        \For{$k = 1$ to $K$}

            \State Assign each particle $i$ a persistent model:
            \Statex \hspace{0.7cm} $b_i \sim \mathrm{Uniform}\{1,\dots,B\}$

            \State Initialize particle state:
            \Statex \hspace{0.7cm} $s_t^{(i)} \gets s_t$

            \For{$h = 0$ to $H-1$}

                \State Extract action $a_{t+h}$ from candidate $A^{(k)}$

                \State Model prediction for particle $i$:
                \Statex \hspace{0.7cm}
                $\mu_{b_i} \gets \mu_{\theta^{(b_i)}}(s_{t+h}^{(i)}, a_{t+h})$
                \Statex \hspace{0.7cm}
                $\Sigma_{b_i} \gets \Sigma_{\theta^{(b_i)}}(s_{t+h}^{(i)}, a_{t+h})$

                \State Sample aleatoric noise:
                \Statex \hspace{0.7cm}
                $\epsilon_{t+h}^{(i)} \sim \mathcal{N}(0,I)$

                \State Next-state prediction:
                \Statex \hspace{0.7cm}
                $s_{t+h+1}^{(i)} = \mu_{b_i} + \sigma_{b_i} \odot \epsilon_{t+h}^{(i)}$

            \EndFor

            \State Predicted return for candidate $A^{(k)}$:
            \Statex \hspace{0.7cm}
            $\hat{J}(A^{(k)}) = \frac{1}{N} \sum_{i=1}^N \sum_{h=0}^{H-1} 
            \gamma^{h} r(s_{t+h}^{(i)}, a_{t+h})$

        \EndFor

        % ---------------------------------------------
        \Statex \textbf{Elite Selection and Distribution Update}

        \State Select top $K_{\text{elite}}$ sequences by $\hat{J}(A^{(k)})$

        \State Compute elite mean:
        \Statex \hspace{0.7cm}
        $m_{\text{elite}} = \frac{1}{K_{\text{elite}}}
        \sum_{A \in \mathcal{E}} A$

        \State Compute elite covariance:
        \Statex \hspace{0.7cm}
        $C_{\text{elite}} = \frac{1}{K_{\text{elite}}}
        \sum_{A \in \mathcal{E}} (A - m_{\text{elite}})(A - m_{\text{elite}})^\top$

        \State Update CEM parameters (smoothed):
        \Statex \hspace{0.7cm}
        $m^{(j+1)} = \alpha m_{\text{elite}} + (1-\alpha)m^{(j)}$
        \Statex \hspace{0.7cm}
        $C^{(j+1)} = \alpha C_{\text{elite}} + (1-\alpha)C^{(j)}$

    \EndFor  % end CEM iterations

    % ---------------------------------------------
    \Statex \textbf{Execute the First Action of Optimized Plan}

    \State Apply first action of optimized sequence:
    \Statex \hspace{0.7cm} $a_t^{*} = m^{(J)}_1$ (Note: $m^{(J)}$  is the best action sequence found by CEM or by trajectory sampling.
    \State Execute $a_t^{*}$ in real environment
    \State Observe $(s_{t+1}, r_{t+1})$
    \State Add $(s_t, a_t^{*}, r_{t+1}, s_{t+1})$ to dataset $\mathcal{D}$

    \If{retraining condition is satisfied}
        \State Retrain ensemble models using bootstrap sampling
    \EndIf

\EndWhile

\end{algorithmic}
\end{algorithm}




\newpage
\pagebreak

\section*{Model-Based Offline Policy Optimization (MOPO) with Conservative Q-Learning (CQL) for Discrete Actions}

MOPO + CQL combines uncertainty-penalized model-based synthetic rollouts (MOPO) with Conservative Q-Learning (CQL), a strong offline RL algorithm for discrete action spaces. CQL explicitly penalizes Q-values for unseen (OOD) actions, preventing overestimation and stabilizing training. This addresses the primary failure mode of offline RL: value extrapolation.

% --------------------------------------------------------------
\subsection*{0. Notation}

\begin{itemize}
    \item $s_t \in \mathcal{S}$: state
    \item $a_t \in \mathcal{A}=\{1,\dots,|\mathcal{A}|\}$: discrete action
    \item $r_{t+1} = r(s_t,a_t)$: offline reward
    \item $s_{t+1}$: next state
    \item $d_{t+1} \in \{0,1\}$: terminal flag
    \item $\mathcal{D}_{\text{offline}}$: fixed offline dataset
    \item $\mathcal{D}_{\text{model}}$: synthetic model transitions
    \item $H_{\text{model}}$: rollout horizon
    \item $\gamma$: discount factor
    \item $\lambda$: uncertainty penalty weight
    \item $E$: ensemble size
    \item $Q_\phi(s,a)$: Q-network
    \item $Q_{\phi^-}(s,a)$: target Q-network
    \item $\alpha$: CQL regularization weight
\end{itemize}

Ensemble dynamics model:
\[
p_{\theta^{(b)}}(s'|s,a)
=
\mathcal{N}\!\left(
\mu_{\theta^{(b)}}(s,a),
\Sigma_{\theta^{(b)}}(s,a)
\right).
\]

Model uncertainty (ensemble disagreement):
\[
u(s,a)
=
\mathrm{Var}_{b=1..E}
\big[\mu_{\theta^{(b)}}(s,a)\big].
\]

% --------------------------------------------------------------
\subsection*{1. Offline Dataset Setup}

\[
\mathcal{D}_{\text{offline}}
=
\{(s_t,a_t,r_{t+1},s_{t+1},d_{t+1})\}.
\]

No real environment interaction occurs:
\[
\text{All data is offline.}
\]

% --------------------------------------------------------------
\subsection*{2. Train Ensemble Dynamics Model}

Each model minimizes negative log-likelihood:
\[
\mathcal{L}_{\text{model}}(\theta^{(b)})
=
-
\mathbb{E}_{(s,a,s')\sim\mathcal{D}_{\text{offline}}}
\left[
\log p_{\theta^{(b)}}(s'|s,a)
\right].
\]

Parameter update:
\[
\theta^{(b)}
\leftarrow
\theta^{(b)}
-
\eta_\theta
\nabla_{\theta^{(b)}}\mathcal{L}_{\text{model}}.
\]

% --------------------------------------------------------------
\subsection*{3. Synthetic Rollout Generation with MOPO Penalty}

Sample start state:
\[
s_0^{(i)} \sim \mathcal{D}_{\text{offline}}.
\]

Sample \textbf{one} model per rollout:
\[
b_i \sim \mathrm{Uniform}\{1,\dots,E\}.
\]

\noindent For each step $h = 0,\dots,H_{\text{model}}-1$:

\paragraph{1. Action selection}
\[
a_h^{(i)} = \arg\max_{a} Q_\phi(s_h^{(i)},a).
\]

\paragraph{2. Next state prediction}
\[
s_{h+1}^{(i)}
=
\mu_{\theta^{(b_i)}}(s_h^{(i)},a_h^{(i)})
+
\xi,
\quad
\xi \sim \mathcal{N}(0,\Sigma_{\theta^{(b_i)}}(s_h^{(i)},a_h^{(i)})).
\]

\paragraph{3. Penalized reward}
\[
\tilde{r}_{h+1}^{(i)}
=
r(s_h^{(i)},a_h^{(i)})
-
\lambda\,u(s_h^{(i)},a_h^{(i)}).
\]

\paragraph{4. Store transition}
\[
(s_h^{(i)}, a_h^{(i)}, \tilde{r}_{h+1}^{(i)}, s_{h+1}^{(i)}, 0)
\in \mathcal{D}_{\text{model}}.
\]

% --------------------------------------------------------------
\subsection*{4. Conservative Q-Learning (CQL)}

Training buffer:
\[
\mathcal{B}
=
\mathcal{D}_{\text{offline}}
\cup
\mathcal{D}_{\text{model}}.
\]

% --------------------
\subsubsection*{4.1 CQL Conservative Penalty (Discrete, Exact)}

\[
\mathcal{L}_{\text{CQL}}^{\text{penalty}}
=
\alpha\,
\mathbb{E}_{s\sim\mathcal{B}}
\left[
\log \sum_{a\in\mathcal{A}} \exp(Q_\phi(s,a))
-
\mathbb{E}_{a\sim\mathcal{B}(a|s)} Q_\phi(s,a)
\right].
\]

% --------------------
\subsubsection*{4.2 TD Target (Double-Q)}

Online action selection:
\[
a^\star = \arg\max_{a'} Q_\phi(s_{t+1},a').
\]

Target value:
\[
y_t =
\tilde{r}_{t+1}
+
\gamma (1-d_{t+1})
Q_{\phi^-}(s_{t+1}, a^\star).
\]

Bellman loss:
\[
\mathcal{L}_{\text{TD}}
=
\mathbb{E}_{(s,a,\tilde{r},s')\sim\mathcal{B}}
\big[
(Q_\phi(s,a)-y_t)^2
\big].
\]

% --------------------
\subsubsection*{4.3 Full CQL Objective}

\[
\mathcal{L}_{\text{CQL}}
=
\mathcal{L}_{\text{TD}}
+
\mathcal{L}_{\text{CQL}}^{\text{penalty}}.
\]

Parameter update:
\[
\phi
\leftarrow
\phi - \eta_\phi \nabla_\phi \mathcal{L}_{\text{CQL}}.
\]

Target update:
\[
\phi^- \leftarrow \phi.
\]

% --------------------------------------------------------------
\subsection*{5. Full MOPO + CQL Algorithm}

\begin{algorithm}[H]
\caption{MOPO + CQL (Discrete Actions)}
\begin{algorithmic}[1]

\State Initialize Q-network $\phi$ and target $\phi^-$
\State Initialize ensemble $\{\theta^{(b)}\}_{b=1}^E$
\State Initialize synthetic buffer $\mathcal{D}_{\text{model}} = \emptyset$

\Statex

\State \textbf{Train ensemble dynamics model}
\For{each model $b$}
    \State minimize NLL on $\mathcal{D}_{\text{offline}}$
\EndFor

\While{not converged}

    \Statex \textbf{Generate synthetic rollouts}
    \For{each rollout $i$}
        \State Sample $s_0^{(i)} \sim \mathcal{D}_{\text{offline}}$
        \State Sample one model index $b_i$
        \For{$h = 0$ to $H_{\text{model}} - 1$}
            \State $a_h^{(i)} = \arg\max_{a} Q_\phi(s_h^{(i)},a)$
            \State Predict $s_{h+1}^{(i)}$ using $\theta^{(b_i)}$
            \State Compute $u(s_h^{(i)},a_h^{(i)})$
            \State Compute $\tilde{r}_{h+1}^{(i)}$
            \State Store transition in $\mathcal{D}_{\text{model}}$
        \EndFor
    \EndFor

    \Statex \textbf{CQL Training}
    \State $\mathcal{B} = \mathcal{D}_{\text{offline}} \cup \mathcal{D}_{\text{model}}$
    \State Compute conservative penalty
    \State Compute TD targets
    \State Update Q-network using full CQL loss
    \State Update target network $\phi^- \leftarrow \phi$

\EndWhile

\end{algorithmic}
\end{algorithm}



\newpage
\pagebreak


\section*{Model-Based Offline Policy Optimization (MOPO) with IQL (Continuous Actions)}

MOPO + IQL for continuous action spaces combines MOPO's ensemble world model 
and uncertainty penalty with IQL's stable offline RL framework and a squashed 
Gaussian policy. Continuous actions are sampled from a $\tanh$-squashed Gaussian, 
ensuring bounded outputs while preserving differentiability.

% ============================================================================
\subsection*{0. Notation}

\begin{itemize}
    \item $s_t \in \mathcal{S}$: state
    \item $a_t \in \mathbb{R}^d$: continuous action
    \item $r_{t+1} = r(s_t,a_t)$: reward from offline dataset
    \item $s_{t+1}$: next state from offline dataset
    \item $d_{t+1} \in \{0,1\}$: termination flag
    \item $\mathcal{D}_{\text{offline}}$: fixed offline dataset
    \item $\mathcal{D}_{\text{model}}$: model-generated transitions
    \item $H_{\text{model}}$: model rollout horizon
    \item $\gamma$: discount factor
    \item $\lambda$: model uncertainty penalty
    \item $E$: ensemble size
    \item $Q_\phi(s,a)$: Q-network
    \item $V_\psi(s)$: expectile value network
    \item $\pi_\theta(a|s)$: squashed Gaussian IQL policy
    \item $\tau \in (0.5,1)$: expectile coefficient
    \item $\beta > 0$: advantage temperature
\end{itemize}

Squashed Gaussian policy:
\[
a = \tanh\!\left( \mu_\theta(s) + \sigma_\theta(s) \epsilon \right),
\qquad
\epsilon \sim \mathcal{N}(0,I).
\]

Ensemble dynamics model:
\[
p_{\theta^{(b)}}(s'|s,a)
=
\mathcal{N}\!\left(
\mu_{\theta^{(b)}}(s,a),\;
\Sigma_{\theta^{(b)}}(s,a)
\right).
\]

Model uncertainty:
\[
u(s,a)
=
\mathrm{Var}_{b=1..E}\!\left[\mu_{\theta^{(b)}}(s,a)\right].
\]

% ============================================================================
\subsection*{1. Offline Dataset Setup}

Offline dataset:
\[
\mathcal{D}_{\text{offline}}
=
\{(s_t,a_t,r_{t+1},s_{t+1},d_{t+1})\}.
\]

Dataset is fixed; no environment interaction occurs:
\[
\text{Real environment is never queried.}
\]

% ============================================================================
\subsection*{2. Train Ensemble Dynamics Model}

Each model minimizes negative log-likelihood:
\[
\mathcal{L}_{\text{model}}(\theta^{(b)})
=
-
\mathbb{E}_{(s,a,s') \sim \mathcal{D}_{\text{offline}}}
\left[
\log p_{\theta^{(b)}}(s'|s,a)
\right].
\]

Gradient update:
\[
\theta^{(b)}
\leftarrow
\theta^{(b)} - \eta_\theta \nabla_{\theta^{(b)}} \mathcal{L}_{\text{model}}.
\]

% ============================================================================
\subsection*{3. Generate Synthetic Rollouts (MOPO)}

Start states:
\[
s_0^{(i)} \sim \mathcal{D}_{\text{offline}}.
\]

Pick ONE world model per rollout:
\[
b_i \sim \mathrm{Uniform}\{1,\dots,E\}.
\]

% -------------------------
Rollout for $h = 0,\dots,H_{\text{model}}-1$:

\paragraph{1. Sample continuous action}
\[
a_h^{(i)}
=
\tanh\!\left(
\mu_\theta(s_h^{(i)})
+
\sigma_\theta(s_h^{(i)})\,\epsilon
\right),
\qquad
\epsilon \sim \mathcal{N}(0,I).
\]

\paragraph{2. Predict next state}
\[
s_{h+1}^{(i)}
=
\mu_{\theta^{(b_i)}}(s_h^{(i)},a_h^{(i)})
+
\xi,
\qquad
\xi \sim \mathcal{N}(0, \Sigma_{\theta^{(b_i)}}(s_h^{(i)},a_h^{(i)})).
\]

\paragraph{3. Penalized reward}
\[
\tilde{r}_{h+1}^{(i)}
=
r(s_h^{(i)},a_h^{(i)}) - \lambda\, u(s_h^{(i)},a_h^{(i)}).
\]

\paragraph{4. Store transition}
\[
(s_h^{(i)},a_h^{(i)},\tilde{r}_{h+1}^{(i)},s_{h+1}^{(i)},0)
\in
\mathcal{D}_{\text{model}}.
\]

% ============================================================================
\subsection*{4. IQL Training (Continuous Actions)}

Training buffer:
\[
\mathcal{B}
=
\mathcal{D}_{\text{offline}}
\cup
\mathcal{D}_{\text{model}}.
\]

% -------------------------
\subsubsection*{4.1 Expectile Value Learning}

\[
V_\psi(s)
=
\arg\min_v
\mathbb{E}_{a\sim D_s}
\left[
L_\tau(Q_\phi(s,a), v)
\right],
\]

Expectile loss:
\[
L_\tau(y,v)
=
|\tau - \mathbf{1}(y < v)| \,(y-v)^2.
\]

% -------------------------
\subsubsection*{4.2 Q-Function Learning}

TD target:
\[
y_t
=
\tilde{r}_{t+1}
+
\gamma(1-d_{t+1}) V_\psi(s_{t+1}).
\]

Q-loss:
\[
\mathcal{L}_Q(\phi)
=
\mathbb{E}_{(s,a,\tilde{r},s')\sim\mathcal{B}}
\left[
(Q_\phi(s,a)-y_t)^2
\right].
\]

% -------------------------
\subsubsection*{4.3 Advantage-Weighted Policy Learning (AWR)}

Implicit advantage:
\[
A(s,a) = Q_\phi(s,a) - V_\psi(s).
\]

Advantage weights:
\[
w(s,a)
=
\exp\!\left(
\frac{A(s,a)}{\beta}
\right).
\]

Policy loss:
\[
\mathcal{L}_\pi(\theta)
=
-
\mathbb{E}_{(s,a)\sim\mathcal{D}_{\text{offline}}}
\left[
w(s,a)\,\log \pi_\theta(a|s)
\right].
\]

% -------------------------
\subsubsection*{4.4 Squashed Gaussian Policy}

Raw Gaussian action:
\[
\tilde{a}
=
\mu_\theta(s)
+
\sigma_\theta(s)\epsilon,
\qquad
\epsilon \sim \mathcal{N}(0,I).
\]

Squashed action:
\[
a = \tanh(\tilde{a}).
\]

Log-probability correction:
\[
\log \pi_\theta(a|s)
=
\log \mathcal{N}(\tilde{a}\, |\, \mu_\theta(s),\sigma_\theta^2(s))
-
\sum_{j=1}^d \log (1 - a_j^2).
\]

% ============================================================================
\subsection*{5. Full MOPO + IQL (Continuous Actions) Training Loop}

\begin{algorithm}[H]
\caption{MOPO + IQL (Continuous Actions)}
\begin{algorithmic}[1]

\State \textbf{Input:} Offline dataset $\mathcal{D}_{\text{offline}}$
\State Initialize ensemble $\{\theta^{(b)}\}_{b=1..E}$
\State Initialize Q-network $\phi$, value network $\psi$
\State Initialize policy parameters $\theta$
\State Initialize synthetic buffer $\mathcal{D}_{\text{model}} \leftarrow \emptyset$

\State \textbf{Train ensemble dynamics} via NLL on $\mathcal{D}_{\text{offline}}$

\While{not converged}

    \State \textbf{Generate synthetic rollouts}
    \For{each rollout $i$}
        \State Sample $s_0^{(i)}$ from $\mathcal{D}_{\text{offline}}$
        \State Sample model index $b_i$
        \For{$h = 0$ \textbf{to} $H_{\text{model}}-1$}
            \State Sample $a_h^{(i)} = \tanh(\mu_\theta(s_h^{(i)}) + \sigma_\theta(s_h^{(i)})\epsilon)$
            \State Predict $s_{h+1}^{(i)}$ via model $\theta^{(b_i)}$
            \State Compute uncertainty $u(s_h^{(i)},a_h^{(i)})$
            \State Compute penalized reward $\tilde{r}_{h+1}^{(i)}$
            \State Store transition in $\mathcal{D}_{\text{model}}$
        \EndFor
    \EndFor

    \State Form buffer $\mathcal{B} = \mathcal{D}_{\text{offline}} \cup \mathcal{D}_{\text{model}}$

    \State Update value network $V_\psi$ via expectile regression
    \State Update Q-network $Q_\phi$ via TD target
    \State Update policy $\pi_\theta$ with advantage-weighted regression

\EndWhile

\end{algorithmic}
\end{algorithm}



\newpage
\pagebreak




% \section*{Two-Level Hierarchical Reinforcement Learning (HRL) with Q-Learning at Both Levels}

% \subsection*{0. Notation}

% \begin{itemize}
%     \item $s_t \in \mathcal{S}$: Environment state at primitive timestep $t$
%     \item $g \in \mathcal{G}=\{g_1,\dots,g_M\}$: Discrete high-level goals
%     \item $a \in \mathcal{A}=\{1,\dots,|\mathcal{A}|\}$: Primitive actions
%     \item $Q^H_{\Phi}(s,g)$: High-level Q-function parameterized by $\Phi$
%     \item $Q^L_{\phi}(s,g,a)$: Low-level Q-function conditioned on goal $g$, parameterized by $\phi$
%     \item Replay buffers: $\mathcal{D}_H$ (high-level), $\mathcal{D}_L$ (low-level)
%     \item $\gamma \in [0,1)$: Discount factor
%     \item $\epsilon_H,\,\epsilon_L$: $\epsilon$-greedy exploration rates for high and low levels
%     \item $\tau(s,g)\in\{0,1\}$: Goal-termination predicate (1 if goal achieved, 0 otherwise)
%     \item $r_{\text{int}}(s, g, s')$: Intrinsic reward function (e.g., $+1$ if $\tau(s',g)=1$, else $0$ or $-\epsilon$)
%     \item Learning rates: $\eta_H$ (high-level), $\eta_L$ (low-level)
%     \item Soft-update rate: $\tau_{\text{target}} \in (0,1]$
%     \item Mini-batch size: $B$
%     \item Warmup steps: $N_{\text{warmup}}$ (minimum buffer size before training)
% \end{itemize}

% \subsection*{1. Initialization}

% Initialize networks:
% \[
% Q^H_{\Phi}, \quad Q^L_{\phi}
% \]
% Initialize target networks:
% \[
% Q^H_{\Phi^-} \gets Q^H_\Phi, 
% \qquad 
% Q^L_{\phi^-} \gets Q^L_\phi
% \]
% Initialize replay buffers:
% \[
% \mathcal{D}_H=\emptyset, \qquad \mathcal{D}_L=\emptyset
% \]

% % ---------------------------------------------------------------------------

% \subsection*{2. Main Training Loop}

% \begin{algorithm}[H]
% \caption{Two-Level Hierarchical Reinforcement Learning (Fully Corrected)}
% \begin{algorithmic}[1]

% \For{each episode}
%     \State Reset environment and observe initial state $s_0$
%     \State $t \gets 0$ \Comment{Global primitive timestep}

%     \While{$t < T_{\max}$ \textbf{and} episode not terminated}

%         % ---------------- HIGH-LEVEL DECISION ---------------------
%         \State \textbf{// High-level goal selection}
%         \State Sample $u \sim \text{Uniform}(0,1)$
%         \If{$u < \epsilon_H$}
%             \State $g_t \sim \text{Uniform}(\mathcal{G})$ \Comment{Explore}
%         \Else
%             \State $g_t \gets \arg\max_{g \in \mathcal{G}} Q^H_\Phi(s_t,g)$ \Comment{Exploit}
%         \EndIf

%         \State $s_{\text{start}} \gets s_t$ \Comment{Initial state for this goal}
%         \State $R^H \gets 0$  \Comment{Accumulated discounted extrinsic return}
%         \State $k \gets 0$    \Comment{Step counter within goal execution}

%         % ---------------- LOW-LEVEL EXECUTION LOOP ----------------
%         \State \textbf{// Execute low-level policy until goal completion}
%         \While{true}
%             \State $s_{\text{current}} \gets s_{t+k}$ \Comment{Current state}
            
%             \State \textbf{// Low-level action selection}
%             \State Sample $u \sim \text{Uniform}(0,1)$
%             \If{$u < \epsilon_L$}
%                 \State $a \sim \text{Uniform}(\mathcal{A})$ \Comment{Explore}
%             \Else
%                 \State $a \gets \arg\max_{a'\in\mathcal{A}} Q^L_\phi(s_{\text{current}},g_t,a')$ \Comment{Exploit}
%             \EndIf

%             \State \textbf{// Execute action in environment}
%             \State Execute $a$ in environment
%             \State Observe reward $r$, next state $s_{\text{next}}$, done flag $d$
            
%             \State \textbf{// Compute intrinsic reward and sub-task termination}
%             \State $\text{goal\_achieved} \gets \tau(s_{\text{next}}, g_t)$ 
%             \State $r_{\text{int}} \gets r_{\text{int}}(s_{\text{current}}, g_t, s_{\text{next}})$ 
%             \State $d_{\text{sub}} \gets d \lor \text{goal\_achieved}$ \Comment{Sub-task done if goal reached or episode ends}

%             \State \textbf{// Store low-level transition}
%             \State $\mathcal{D}_L \gets \mathcal{D}_L \cup \{(s_{\text{current}},g_t,a,r_{\text{int}},s_{\text{next}},d_{\text{sub}})\}$

%             \State \textbf{// Accumulate high-level discounted extrinsic reward}
%             \State $R^H \gets R^H + \gamma^{k} \cdot r$

%             \State \textbf{// Low-level Q-learning update}
%             \If{$|\mathcal{D}_L| \ge N_{\text{warmup}}$}
%                 \State Sample mini-batch $\mathcal{B}_L$ from $\mathcal{D}_L$
%                 \State Perform low-level update (Section 3)
%             \EndIf

%             \State $k \gets k + 1$ \Comment{Increment step counter}

%             \State \textbf{// Check termination conditions}
%             \If{$d = 1$ \textbf{ or } $\text{goal\_achieved} = 1$}
%                 \State $s_{\text{end}} \gets s_{\text{next}}$ \Comment{Final state after goal execution}
%                 \State $d_{\text{end}} \gets d$ \Comment{Environment termination flag}
%                 \State $T_g \gets k$ \Comment{Total steps taken for this goal}
%                 \State \textbf{break} \Comment{Exit low-level loop}
%             \EndIf

%         \EndWhile

%         % ---------------- HIGH-LEVEL TRANSITION ----------------
%         \State \textbf{// Store high-level transition}
%         \State $\mathcal{D}_H \gets \mathcal{D}_H \cup \{(s_{\text{start}}, g_t, R^H, s_{\text{end}}, d_{\text{end}}, T_g)\}$

%         \State \textbf{// High-level Q-learning update}
%         \If{$|\mathcal{D}_H| \ge N_{\text{warmup}}$}
%             \State Sample mini-batch $\mathcal{B}_H$ from $\mathcal{D}_H$
%             \State Perform high-level update (Section 4)
%         \EndIf

%         \State $t \gets t + T_g$ \Comment{Advance global timestep by goal duration}

%         \If{$d_{\text{end}} = 1$}
%             \State \textbf{break} \Comment{Episode terminated}
%         \EndIf

%     \EndWhile
% \EndFor

% \end{algorithmic}
% \end{algorithm}

% % ---------------------------------------------------------------------------

% \subsection*{3. Low-Level Q-Learning Update}

% Sample mini-batch:
% \[
% \mathcal{B}_L=\{(s^{(i)},g^{(i)},a^{(i)},r_{\text{int}}^{(i)},s'^{(i)},d_{\text{sub}}^{(i)})\}_{i=1}^B \sim \mathcal{D}_L
% \]

% Compute TD targets using \textbf{intrinsic rewards} and \textbf{sub-task termination}:
% \[
% y_i^L = r_{\text{int}}^{(i)} + \gamma(1-d_{\text{sub}}^{(i)}) \max_{a'} Q^L_{\phi^-}(s'^{(i)}, g^{(i)}, a')
% \]

% \textbf{Key properties:}
% \begin{itemize}
%     \item $r_{\text{int}}^{(i)}$ provides goal-oriented feedback, solving the sparse reward problem
%     \item $d_{\text{sub}}^{(i)} = 1$ when goal is achieved \textbf{or} environment terminates
%     \item Setting $d_{\text{sub}} = 1$ upon goal achievement prevents bootstrapping from states beyond the goal boundary, ensuring proper value estimation for the sub-task
% \end{itemize}

% Compute loss:
% \[
% \mathcal{L}^L(\phi)=\frac{1}{B}\sum_{i=1}^B \left(Q^L_\phi(s^{(i)},g^{(i)},a^{(i)})-y_i^L\right)^2
% \]

% Gradient update:
% \[
% \phi \gets \phi - \eta_L \nabla_\phi \mathcal{L}^L(\phi)
% \]

% Soft target network update:
% \[
% \phi^- \gets \tau_{\text{target}}\phi + (1-\tau_{\text{target}})\phi^-
% \]

% % ---------------------------------------------------------------------------

% \subsection*{4. High-Level Q-Learning Update (SMDP Formulation)}

% Sample mini-batch:
% \[
% \mathcal{B}_H=\{(s^{(i)},g^{(i)},R^{(i)},s'^{(i)},d^{(i)},T_g^{(i)})\}_{i=1}^B \sim \mathcal{D}_H
% \]

% Compute TD targets with \textbf{goal-duration discounting} (Semi-Markov Decision Process):
% \[
% y_i^H = R^{(i)} + \gamma^{T_g^{(i)}}(1-d^{(i)}) \max_{g'} Q^H_{\Phi^-}(s'^{(i)},g')
% \]

% \textbf{Rationale:} 
% The return $R^{(i)}$ is computed as:
% \[
% R^{(i)} = \sum_{\ell=0}^{T_g^{(i)}-1} \gamma^\ell r_{t+\ell}
% \]
% This is the discounted sum of extrinsic rewards over $T_g^{(i)}$ primitive timesteps. To maintain temporal consistency in the Bellman equation, the bootstrap value at the next high-level state $s'^{(i)}$ must be discounted by $\gamma^{T_g^{(i)}}$, accounting for the time elapsed during goal execution.

% Compute loss:
% \[
% \mathcal{L}^H(\Phi)=\frac{1}{B}\sum_{i=1}^B 
% \left(Q^H_\Phi(s^{(i)},g^{(i)})-y_i^H\right)^2
% \]

% Gradient update:
% \[
% \Phi \gets \Phi - \eta_H \nabla_\Phi \mathcal{L}^H(\Phi)
% \]

% Soft target network update:
% \[
% \Phi^- \gets \tau_{\text{target}}\Phi + (1-\tau_{\text{target}})\Phi^-
% \]

% % ---------------------------------------------------------------------------

% \subsection*{5. Summary of Key Design Principles}

% \begin{enumerate}
%     \item \textbf{Temporal Abstraction (SMDP):} The high-level policy operates in extended time by selecting goals that persist for variable durations. The Q-function update accounts for this via $\gamma^{T_g}$ discounting.
    
%     \item \textbf{Dual Reward Structure:}
%     \begin{itemize}
%         \item \textbf{Low-level:} Uses intrinsic rewards $r_{\text{int}}(s,g,s')$ to learn goal-conditioned behavior
%         \item \textbf{High-level:} Uses extrinsic environment rewards $r$ to learn which goals lead to task success
%     \end{itemize}
    
%     \item \textbf{Goal-Aware Termination:} Low-level transitions use $d_{\text{sub}} = d_{\text{env}} \lor \tau(s',g)$ to treat goal achievement as a terminal state for the sub-task, preventing value contamination.
    
%     \item \textbf{Hierarchical Exploration:} Independent $\epsilon$-greedy exploration at both levels allows the hierarchy to discover both useful goals and effective low-level policies.
    
%     \item \textbf{Decoupled Learning:} Separate replay buffers and target networks for each level enable stable off-policy learning at different temporal scales.
% \end{enumerate}

% % ---------------------------------------------------------------------------

% \subsection*{6. Common Intrinsic Reward Functions}

% The choice of $r_{\text{int}}(s, g, s')$ significantly impacts low-level learning:

% \begin{itemize}
%     \item \textbf{Binary sparse:} 
%     \[
%     r_{\text{int}}(s,g,s') = \begin{cases} +1 & \text{if } \tau(s',g)=1 \\ 0 & \text{otherwise} \end{cases}
%     \]
%     Simple but provides no gradient for learning.
    
%     \item \textbf{Binary with step penalty:} 
%     \[
%     r_{\text{int}}(s,g,s') = \begin{cases} +1 & \text{if } \tau(s',g)=1 \\ -0.01 & \text{otherwise} \end{cases}
%     \]
%     Encourages faster goal completion.
    
%     \item \textbf{Distance-based (continuous goals):} 
%     \[
%     r_{\text{int}}(s,g,s') = -\|f(s') - f(g)\|_2
%     \]
%     where $f: \mathcal{S} \cup \mathcal{G} \to \mathbb{R}^d$ embeds states and goals in a metric space. Provides dense learning signal.
    
%     \item \textbf{Shaped reward:}
%     \[
%     r_{\text{int}}(s,g,s') = \begin{cases} 
%     +10 & \text{if } \tau(s',g)=1 \\
%     \gamma \Phi(s',g) - \Phi(s,g) & \text{otherwise}
%     \end{cases}
%     \]
%     where $\Phi(s,g)$ is a potential function (e.g., negative distance to goal). This is theoretically equivalent to sparse rewards under certain conditions.
% \end{itemize}

% % ---------------------------------------------------------------------------


\newpage
\pagebreak

% \section*{Generic Curriculum Learning for Discrete-Action Q-Learning (CL--DQL)}
% Curriculum Learning (CL) gradually increases the difficulty of tasks presented to an RL agent.  
% The agent is trained first on simpler tasks and then on progressively harder ones, improving sample efficiency, stability, and convergence.

% This framework applies specifically to \textbf{discrete action spaces} using Q-learning methods.

% \subsection*{0. Notation}

% \begin{itemize}
%     \item $s_t \in \mathcal{S}$: environment state at timestep $t$
%     \item $a_t \in \mathcal{A}$: action at timestep $t$ (where $\mathcal{A}$ is a \textbf{finite, discrete} action set)
%     \item $r_{t+1}$: reward received after taking $a_t$
%     \item $s_{t+1}$: next state
%     \item $d_{t+1}\in\{0,1\}$: termination flag entering $s_{t+1}$
%     \item $\mathcal{T}=\{T_1,\dots,T_K\}$: a sequence of tasks ordered by difficulty
%     \item $T_k$: the task at curriculum level $k$ ($1 =$ easiest)
%     \item $Q_\phi(s,a)$: Q-network
%     \item $Q_{\phi^-}(s,a)$: target Q-network
%     \item $\epsilon$: exploration rate
%     \item $\gamma\in[0,1)$: discount factor
%     \item $\eta$: learning rate
%     \item $\mathcal{D}$: replay buffer
%     \item $B$: mini-batch size
%     \item $\tau_{\text{target}}$: soft-update rate for target network
%     \item $P_{\text{progress}}(T_k)$: curriculum progression criterion for task $T_k$
%           (e.g., average return threshold, success rate threshold)
% \end{itemize}

% \subsection*{1. Initialization}

% Initialize Q-network: \[Q_\phi(s,a)\]
% Initialize target network: \[Q_{\phi^-}\gets Q_\phi\]
% Initialize replay buffer:\[\mathcal{D}=\emptyset\]
% Set curriculum index:\[k\gets 1\]
% (i.e., start on the easiest task $T_1$)

% % ---------------------------------------------------------------------------

% \subsection*{2. Main Training Loop with Curriculum Progression}

% \begin{algorithm}[H]
% \caption{Curriculum Learning for Discrete-Action Q-Learning}
% \begin{algorithmic}[1]

% \State $k\gets 1$ \Comment{Start with easiest task $T_1$}

% \While{$k \le K$} \Comment{Proceed through curriculum levels}

%     \For{each episode under task $T_k$}
%         \State Reset environment for task $T_k$ and observe $s_0$
%         \State $t\gets 0$

%         \While{$t < T_{\max}$}

%             % ---------------- ACTION SELECTION (DISCRETE) ----------------
%             \State Sample $u \sim \text{Uniform}(0,1)$
%             \If{$u < \epsilon$}
%                 \State $a_t \sim \text{Uniform}(\mathcal{A})$ \Comment{Explore: random discrete action}
%             \Else
%                 \State $a_t \gets \arg\max_{a\in\mathcal{A}} Q_\phi(s_t,a)$ \Comment{Exploit: greedy over discrete actions}
%             \EndIf

%             % ---------------- TRANSITION ----------------
%             \State Execute $a_t$ under task $T_k$
%             \State Observe $r_{t+1}$, $s_{t+1}$, $d_{t+1}$

%             % ---------------- STORE TRANSITION ----------------
%             \State $\mathcal{D}\gets \mathcal{D} \cup \{(s_t,a_t,r_{t+1},s_{t+1},d_{t+1})\}$

%             % ---------------- Q-LEARNING UPDATE ----------------
%             \If{$|\mathcal{D}|\ge B$}
%                 \State Sample mini-batch $\mathcal{B}$ of size $B$ from $\mathcal{D}$

%                 \For{each transition $(s^{(i)},a^{(i)},r^{(i)},s'^{(i)},d^{(i)})\in\mathcal{B}$}
%                     \State Compute target (using discrete max):
%                     \[
%                     y_i = r^{(i)} + \gamma(1-d^{(i)})\max_{a'\in\mathcal{A}}Q_{\phi^-}(s'^{(i)},a')
%                     \]
%                 \EndFor

%                 \State Compute loss:
%                 \[
%                 \mathcal{L}(\phi)=\frac{1}{B}\sum_{i=1}^B\left(Q_\phi(s^{(i)},a^{(i)})-y_i\right)^2
%                 \]

%                 \State Update Q-network:
%                 \[
%                 \phi \gets \phi - \eta\nabla_\phi \mathcal{L}(\phi)
%                 \]

%                 \State Soft-update target network:
%                 \[
%                 \phi^- \gets \tau_{\text{target}}\phi + (1-\tau_{\text{target}})\phi^-
%                 \]
%             \EndIf

%             % ---------------- TIME UPDATE ----------------
%             \State $t\gets t+1$

%             \If{$d_{t}=1$}
%                 \State \textbf{break}
%             \EndIf

%         \EndWhile

%         % ---------------- CURRICULUM PROGRESSION CHECK ----------------
%         \If{$P_{\text{progress}}(T_k)=1$}
%             \State $k \gets k+1$  \Comment{Move to next task in curriculum}
%             \State \textbf{break}
%         \EndIf

%     \EndFor

% \EndWhile

% \end{algorithmic}
% \end{algorithm}

% % ---------------------------------------------------------------------------

% \subsection*{3. Curriculum Progression Criteria (Generic $P_{\text{progress}}$)}

% A task $T_k$ is considered ``mastered'' when:
% \[
% P_{\text{progress}}(T_k)=
% \begin{cases}
% 1, & \text{if success-rate}(T_k) \ge \alpha, \\[6pt]
% 1, & \text{if average episodic return} \ge R_{\text{threshold}},\\[6pt]
% 1, & \text{if agent solves the task consistently for $N$ episodes},\\[6pt]
% 0, & \text{otherwise}.
% \end{cases}
% \]

% Examples of valid progression signals:
% \begin{itemize}
%     \item Reaching a goal location at least $\alpha\%$ of episodes.
%     \item Solving a maze of current difficulty consistently.
%     \item Achieving a threshold return.
%     \item Convergence of TD updates (small Bellman error).
% \end{itemize}

% % ---------------------------------------------------------------------------


\newpage
\pagebreak




% ---------------------------------------------------------------------------

% \section*{Generic Multi-Agent Reinforcement Learning (Discrete Action Space)}

% We consider a fully generic multi-agent Markov game with $N$ agents, discrete actions,
% decentralized execution, and optionally centralized critics. The formulation is suitable
% for cooperative, competitive, or mixed multi-agent environments.

% % ---------------------------------------------------------------------------

% \subsection*{0. Notation (Short, Complete)}

% \begin{itemize}
%     \item $N$: number of agents.
%     \item $s_t \in \mathcal{S}$: global state at timestep $t$.
%     \item $\mathcal{O}^{(i)}(s_t)$: observation available to agent $i$ at state $s_t$.
%     \item $a_t^{(i)} \in \mathcal{A}^{(i)}$: discrete action of agent $i$.
%     \item $a_t = (a_t^{(1)},\dots,a_t^{(N)})$: joint action.
%     \item $r_{t+1}^{(i)}$: reward of agent $i$ at time $t+1$.
%     \item $s_{t+1}$: next state.
%     \item $d_{t+1} \in \{0,1\}$: termination flag.
%     \item $\pi_{\theta_i}(a^{(i)} \mid \mathcal{O}^{(i)}(s))$: decentralized policy of agent $i$.
%     \item $Q_{\phi_i}(s, a^{(i)}, a^{-i})$: centralized critic for agent $i$ (CTDE).
%     \item $V_{\psi_i}(s)$: value function of agent $i$.
%     \item $\gamma$: discount factor.
%     \item Replay buffer: $\mathcal{D}$ containing full joint transitions.
% \end{itemize}

% Joint dynamics:
% \[
% s_{t+1} \sim p(s_{t+1} \mid s_t, a_t).
% \]

% Objective for each agent:
% \[
% J(\pi_{\theta_i}) = 
% \mathbb{E}\left[\sum_{t=0}^{T-1} \gamma^t r_{t+1}^{(i)}\right].
% \]

% % ---------------------------------------------------------------------------

% \subsection*{1. Decentralized Actors with Optional Centralized Critics (CTDE)}

% \paragraph{Actors (execution):}
% Each agent acts independently:
% \[
% a_t^{(i)} \sim \pi_{\theta_i}(\cdot \mid \mathcal{O}^{(i)}(s_t)).
% \]

% \paragraph{Centralized critic (training only):}
% \[
% Q_{\phi_i}(s_t, a_t^{(i)}, a_t^{-i})
% \]
% can use the full state and/or full joint action for stable training.

% During execution, only $\pi_{\theta_i}$ is used.

% % ---------------------------------------------------------------------------

% \subsection*{2. Bellman Target (Per-Agent)}

% For each agent $i$, the target for critic learning is:
% \[
% y^{(i)} 
% =
% r_{t+1}^{(i)}
% + \gamma (1 - d_{t+1})
% \max_{a'^{(i)}} Q_{\phi_i^-}(s_{t+1}, a'^{(i)}, a_{t+1}^{-i}).
% \]

% Here:
% \[
% a_{t+1}^{-i} = (a_{t+1}^{(1)},\dots,a_{t+1}^{(i-1)},a_{t+1}^{(i+1)},\dots,a_{t+1}^{(N)})
% \]
% comes from the next-step joint action under current policies.

% % ---------------------------------------------------------------------------

% \subsection*{3. Critic Loss (Per-Agent)}

% \[
% \mathcal{L}^{(i)}_{\text{critic}}
% =
% \mathbb{E}_{(s,a,r^{(i)},s',d) \sim \mathcal{D}}
% \Big[
% \left(Q_{\phi_i}(s, a^{(i)}, a^{-i}) - y^{(i)}\right)^{2}
% \Big].
% \]

% This is a standard TD(0) loss with joint information used for stability.

% % ---------------------------------------------------------------------------

% \subsection*{4. Policy Gradient (Decentralized Execution)}

% The decentralized actor is updated using advantage-weighted gradients:

% \[
% \nabla_{\theta_i} J(\pi_{\theta_i})
% =
% \mathbb{E}
% \Big[
% \nabla_{\theta_i} \log \pi_{\theta_i}(a^{(i)} \mid \mathcal{O}^{(i)}(s))
% \cdot A^{(i)}(s, a^{(i)})
% \Big].
% \]

% Advantage for agent $i$:
% \[
% A^{(i)}(s,a^{(i)})
% =
% Q_{\phi_i}(s, a^{(i)}, a^{-i})
% -
% V_{\psi_i}(s).
% \]

% Value function:
% \[
% V_{\psi_i}(s)
% =
% \mathbb{E}_{a^{(i)}\sim\pi_{\theta_i}}
% \left[ 
% Q_{\phi_i}(s, a^{(i)}, a^{-i})
% \right].
% \]

% % ---------------------------------------------------------------------------

% \subsection*{5. Value Function Learning (Optional)}

% Value loss:
% \[
% \mathcal{L}^{(i)}_V
% =
% \mathbb{E}
% \left[
% \big(V_{\psi_i}(s_t) - R_t^{(i)}\big)^2
% \right],
% \]
% where
% \[
% R_t^{(i)} = \sum_{l=0}^{T-t-1}\gamma^l r_{t+l+1}^{(i)}.
% \]

% % ---------------------------------------------------------------------------

% \subsection*{6. Multi-Agent Transition Dataset}

% Replay buffer contains full joint information:
% \[
% \mathcal{D}
% =
% \Big\{
% (s_t,
% a_t^{(1)},\dots,a_t^{(N)},
% r_{t+1}^{(1)},\dots,r_{t+1}^{(N)},
% s_{t+1},
% d_{t+1})
% \Big\}.
% \]

% Full-state and full-joint-action storage enables CTDE.

% % ---------------------------------------------------------------------------

% \subsection*{7. Training Updates (Per-Agent)}

% \paragraph{Critic Update:}
% \[
% \phi_i \gets \phi_i - \eta_\phi \nabla_{\phi_i} 
% \mathcal{L}^{(i)}_{\text{critic}}.
% \]

% \paragraph{Actor Update:}
% \[
% \theta_i \gets \theta_i + \eta_\theta \nabla_{\theta_i} J(\pi_{\theta_i}).
% \]

% \paragraph{Target Network Update:}
% \[
% \phi_i^{-} \gets \tau \phi_i + (1-\tau)\phi_i^{-}.
% \]

% These are repeated for each agent $i=1,\dots,N$.

% % ---------------------------------------------------------------------------

% \subsection*{8. Optional: Value Factorization for Cooperative MARL}

% In fully cooperative settings, a factored value function can be used:

% \[
% Q_{\text{tot}}(s,a^{(1)},\dots,a^{(N)})
% = f_{\text{mix}}(Q_1,\dots,Q_N),
% \]

% where $f_{\text{mix}}$ is constrained to be monotonic (QMIX) or additive (VDN).

% This enables coordinated learning while keeping decentralized execution.

% % ---------------------------------------------------------------------------

% \subsection*{9. Decentralized Execution}

% At deployment time, agents act independently using only their own observations:

% \[
% a_t^{(i)} = 
% \arg\max_{a} \pi_{\theta_i}(a \mid \mathcal{O}^{(i)}(s_t)).
% \]

% Critics and joint state information are not required at execution time.

% % ---------------------------------------------------------------------------

% \subsection*{10. Summary (Fully Generic)}

% \begin{enumerate}
%     \item \textbf{CTDE}: centralized training with decentralized execution.
%     \item \textbf{Discrete policies}: each agent uses $\pi_{\theta_i}(a\mid o)$.
%     \item \textbf{Critics}: can depend on global state and joint actions.
%     \item \textbf{Individual advantage calculation}: ensures correct gradients.
%     \item \textbf{Terminal masking}: $(1-d_{t+1})$ prevents bootstrap at terminal states.
%     \item \textbf{Fully generic}: applies to cooperative, competitive, or mixed settings.
%     \item \textbf{No off-by-one errors}: all transitions use $(s_t,a_t,r_{t+1},s_{t+1},d_{t+1})$.
% \end{enumerate}

% 4. Latent Dynamical Models (World Models)

% These learn compact latent state spaces → plan in imagination.

% �� 4.1. World Models (Ha & Schmidhuber)

% Learn VAE + MDN-RNN latent dynamics.

% Plan using CMA-ES or random shooting.

% �� 4.2. PlaNet

% Latent dynamics via Recurrent State-Space Model (RSSM).

% Planning using Cross-Entropy MPC.

% �� 4.3. Dreamer (V1/V2/V3)

% Actor-critic entirely inside the latent world.

% Learn reward, transition, and value all in latent space.

% DreamerV3 = state-of-the-art general agent.


% �� 8.1. DreamerV3

% Breakthrough general world-model agent.

% �� 8.2. IRIS / GenDP / Dyna-Tree

% Diffusion models for dynamics modeling.

% �� 8.3. MDP Transformers / Decision Transformers + Models

% Sequence models predicting trajectories + planning.

% �� 8.4. SMAC Zero / AlphaZero for Multi-Agent Systems

% Deep planning + latent models.

\end{document}






